\documentclass[opermanual]{espd}
\usepackage[russian]{babel}

\bibliographystyle{gost2008}

\managerrank{Научный руководитель,\\доцент департамента\\программной инженерии\\факультета компьютерных наук,\\канд. техн. наук}
\manager{С.Л. Макаров}

\authorrank{студент группы БПИ183}
\author{М.И. Сердюков}

\title{Мессенджер с поиском по Bluetooth\\(клиентская часть)}
\code{02.07}

\city{Москва}
\year{2021}

\begin{document}

\tableofcontents

\section{Назначение программы}
Программа (мобильное приложение) предназначена для отправки и приема сообщений между пользователями. 

\section{Условия выполнения программы}
\subsection{Требования к аппаратным средствам}
Для корректной работы приложения необходимо мобильное устройство со следующими техническими минимальными характеристиками:
\begin{enumerate}
\item 500 мб RAM;
\item 1 Гб дискового пространства;
\item Доступ в интернет;
\item Наличие bluetooth модуля.
\end{enumerate}

\subsection{Требования к программным средствам}
Мобильное устройство должно иметь операционную систему Android 7 или выше. 

\section{Запуск программы}
Далее будет описано, как корректно установить и запустить программу на мобильном устройстве.

\subsection{Установка}
После загрузки установочного .apk архива, необходимо открыть меню файловой сестемы "Загрузки" и тапнуть на скаченный архив. Начнется установка приложения.

\subsection{Запуск}
Запуск происходит на мобильном устройстве. После установки, в списке всех приложений должно появится приложение с названием closecom, после нажатия на него, оно откроется.

\section{Использование приложения}

\subsection{Авторизация}

\illustration[][][0.3]{img/auth}[auth]

Создание аккаунта и авторизация происходит на экране авторизации (\ref{auth}).
От пользователя требуется ввести email и пароль затем нажать кнопку-действие.

\illustration[][][0.3]{img/auth_email_password}[auth_email_password]

Введем email и пароль (\ref{auth_email_password}). Затем нажмем кнопку регистрации "Sign up". При условии, что данные введены верно, мы должны попасть на главный экран чатов.

\illustration[][][0.3]{img/chats_not_empty}[chats_not_empty]

В результате авторизации мы попали на экран чатов (\ref{chats_not_empty}), пока что он пустой.

\subsection{Поиск пользователей по email}

\illustration[][][0.3]{img/search_users}[search_users]

Поиск пользователей по индентификатору (email) происходит на экране поиска по email (\ref{search_users}). Введем запрос в поисковую строку вверху экрана.

\illustration[][][0.3]{img/search_user_result}[search_user_result]

В списке появится результат поиска (\ref{search_user_result}). Нажав на имя пользователя мы попадем на экран сообщений и сможем написать ему.

\subsection{Поиск пользователей по bluetooth}

\illustration[][][0.3]{img/bluetooth_search}[bluetooth_search]

Для поиска по bluetooth зайдем на экран поиска по блютуз (\ref{bluetooth_search}). После того как пользователь заходит в этот экран начинается сканирование пользователей вокруг. Формируется список пользователей найденных поблизости.

\subsection{Отправка сообщений}

При нажатии на имя пользователя из списка результатов поиска (\ref{search_user_result}), или при нажатии на чат на экране списков чатов (\ref{chats_not_empty}) откроется экран списка сообщений, в нижней части которого расположено поле для ввода и отправки сообщений.

\illustration[][][0.3]{img/messaging_type}[messaging_type]

Пример сообщения, введенного пользователем (\ref{messaging_type}).

\illustration[][][0.3]{img/messaging_sent}[messaging_sent]

После нажатия на кнопку отправить (\ref{messaging_sent}), сообщение отправится собеседнику.

\section{Сообщения оператору}

\subsection{Показ ошибок}

На экране авторизации (\ref{auth}), при вводе некорректных данных email или пароля появится диалоговое окно с ошибкой. 

\illustration[][][0.3]{img/auth_error}[auth_error]

Таким образом осуществляется оповещение пользователей о неправильном вводе данных.  
 
\end{document}

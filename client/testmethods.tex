\documentclass[testmethods]{espd}
\usepackage[russian]{babel}

\bibliographystyle{gost2008}

\managerrank{Научный руководитель,\\доцент департамента\\программной инженерии\\факультета компьютерных наук,\\канд. техн. наук}
\manager{С.Л. Макаров}

\authorrank{студент группы БПИ183}
\author{М.И. Сердюков}

\title{Мессенджер с поиском по Bluetooth\\(клиентская часть)}
\code{02.07}

\city{Москва}
\year{2021}

\begin{document}

\tableofcontents

\section{Объект испытаний}
\subsection{Наименование программы}
\paragraph{Наименование программы на русском языке}
Мессенджер с поиском по Bluetooth.
\paragraph{Наименование программы на английском языке}
Messenger with Bluetooth Search.

\subsection{Область применения}
Программа предназначена для передачи сообщений в виде текста и файлов, между пользователями.

Программа применяется как клиентаская часть общего продукта, вместе с сервером.

\subsection{Обозначение испытуемой программы}
Краткое обозначение испытуемой программы -- <<closecom>>.

\section{Цель испытаний}
\subsection{Цель проведения испытаний}
Целью испытаний является проверка заявленных в Техническом задании требований к функциональности и надежности мобильного приложения. Проведение испытаний обосновано в Техническом задании.

\section{Требования к программе}
\subsection{Требования к функциональным характеристикам}

Программа (мобильное приложение) должна обеспечивать следующий функционал:

\begin{enumerate}
\item Авторизация в приложении через электронную почту и пароль (создание аккаунта, регистрация);
\item Поиск пользователей в приложении по его идентификатору;
\item Возможность написать сообщение другому пользователю приложения (в виде текста и файлов);
\item Возможность просмотра истории сообщений с другим пользователем (чат с пользователем);
\item Возможность удаления чата с другим пользователем;
\item Начать переписку с другим пользователем "поблизости", используя Bluetooth;
\item Изменить электронную почту и пароль, привязанные к аккаунту;
\item Пользователь должен иметь возможность удалить собственный аккаунт.
\end{enumerate}

\paragraph{Требования к интерфейсу}

Мобильное приложение должно иметь следующую структуру интерфейса:

\begin{enumerate}
\item Экран авторизации;
\item Экран со списком всех доступных чатов;
\item Меню с настройками аккаунта и приложения;
\item Список всех сообщений с конкретным пользователем (экран чата);
\item Экран поиска пользователей по идентификатору;
\item Экран пользователей "поблизости", для поиска с использованием Bluetooth.
\end{enumerate}

\paragraph{Требования к организации входных данных}
Входными данными для программы служат данные из сети интернет, Bluetooth, а также ввод текста и файлов с устройства пользователя.

\paragraph{Требования к организации выходных данных}
В качестве выходных данных используются файлы и тексты, передаваемые по интернету серверному приложению, вывод информации на экран пользователя и Bluetooth данные.

\subsection{Требования к надежности}
Программа должна проверять входные данные и корректно работать при любых сценариях пользователя.
\subsection{Условия эксплуатации}
Климатические условия эксплуатации программы определяются климатическими условиями эксплуатации оборудования, используемого для хранения и запуска программы. Таким образом, должны выполняться следующие условия:

\begin{enumerate}
\item Влажность -- не более 80\%;
\item Температура -- от 10 $^\circ$C до 30 $^\circ$C;
\item Атмосферное давление от 630 до 800 мм рт. ст.;
\item Отсутствие газообразных кислот и коррозийных веществ в воздухе;
\item Запыленность не более 0.75 мг/м$^3$.
\end{enumerate}

\subsection{Требования к составу и параметрам технических средств}\label{subsection:requirements}
Для работы программы необходимо мобильное устройство на платформе Android (смартфон). Минимальные системные требования:

\begin{enumerate}
\item Оперативная память объемом 1 Гб;
\item Размер внутренней памяти 500 Мб;
\item Наличие Bluetooth модуля;
\item Наличие подключения к сети интернет;
\end{enumerate}

\subsection{Требования к информационной и программной совместимости}
Для работы программы требуется наличие операционной системы Android, с Api не менее 24 версии (Android 7.0 и более).

\subsection{Требования к маркировке и упаковке}
Программа поставляется в виде .apk архива.

\subsection{Требования к транспортированию и хранению}
Требования к транспортировке и хранению программных документов являются стандартными и должны соответствовать общим требованиям хранения и транспортировки печатной продукции:

\begin{enumerate}
\item В помещении для хранения печатной продукции допустимы температура воздуха от 10 $^\circ$С до 30 $^\circ$С и относительная влажность воздуха от 30\% до 60\%;
\item Документацию хранят и используют на расстоянии не менее 0.5 м от источников тепла и влаги. Не допускается хранение печатной продукции в помещениях, где находятся агрессивные агенты – растворители, спирт, бензин;
\item Не допускается попадание на документацию агрессивных агентов;
\item Транспортировка производится в специальных контейнерах с применением мер по предотвращению деформации документов внутри контейнеров, а также проникновения влаги, вредных газов, пыли, солнечных лучей и образованию конденсата внутри контейнеров;
\item Программные документы, предоставляемые в печатном виде, должны соответствовать общим правилам учета и хранения программных документов, предусмотренных стандартами Единой системы программной документации и соответствовать требованиям ГОСТ 19.602-78~\cite{espd602}.
\end{enumerate}

\section{Требования к программной документации}
\subsection{Предварительный состав программной документации}\label{subsection:documentation}
<<Мессенджер с поиском по Bluetooth>>. Техническое задание (ГОСТ 19.201-78~\cite{espd201})

<<Мессенджер с поиском по Bluetooth>>. Программа и методика испытаний (ГОСТ 19.301-78~\cite{espd301})

<<Мессенджер с поиском по Bluetooth>>. Пояснительная записка (ГОСТ 19.404-79~\cite{espd404})

<<Мессенджер с поиском по Bluetooth>>. Руководство оператора (ГОСТ 19.505-79~\cite{espd505})

<<Мессенджер с поиском по Bluetooth>>. Текст программы (ГОСТ 19.401-78~\cite{espd401})

\subsection{Специальные требования к программной документации}\label{subsection:docspec}
Документы к программе должны быть выполнены в соответствии с ГОСТ 19.106-78~\cite{espd106} и ГОСТами к каждому виду документа (см. п.~\ref{subsection:documentation}).

Пояснительная записка должна быть загружена в систему <<Антиплагиат>> через LMS НИУ ВШЭ.

Документация и программа сдаются в электронном виде в формате .pdf или .docx в архиве формата .zip или .rar.

За один день до защиты комиссии все материалы курсового проекта:
\begin{enumerate}
\item Техническая документация;
\item Программный проект;
\item Исполняемый файл;
\item Отзыв руководителя;
\item Лист Антиплагиата.
\end{enumerate}
должны быть загружены одним или несколькими архивами в проект дисциплины <<Курсовой проект 2020-2021>> в личном кабинете информационной образовательной среде LMS (Learning Management System) НИУ ВШЭ.

\section{Средства и порядок испытаний}

\subsection{Порядок проведения испытаний}
\begin{enumerate}
\item Проверка выполнения требований к интерфейсу;
\item Проверка выполнения к функциональным требованиям;
\end{enumerate}

\section{Методы испытаний}
\subsection{Описание используемых методов испытаний для проверки программной документации}
Документация проверяется на соответствие составу (см. п.~\ref{subsection:documentation}). Также должно быть соответствие стандартам ГОСТ 19 ЕСПД, указанным в этом пункте.

\subsection{Требования к интерфейсу}

\paragraph{} %1

\illustration[][][0.3]{img/auth}[auth]

На (\ref{auth}) изображен экран авторизации.

\paragraph{} %2

\illustration[][][0.3]{img/chats_not_empty}[chats_not_empty]

На экране (\ref{chats_not_empty}) представлен список всех чатов пользователя.

\paragraph{} %3

\illustration[][][0.3]{img/settings}[settings]

Экран (\ref{settings}) с настройками приложения.

\paragraph{} %4

\illustration[][][0.3]{img/messaging_result}[messaging_result]

Экран со списком (\ref{messaging_result}) сообщений с конкретным пользователем.

\paragraph{} %5

\illustration[][][0.3]{img/search_users}[search_users]

Экран поиска пользователей по email (\ref{search_users}).

\paragraph{} %6

\illustration[][][0.3]{img/bluetooth_search}[bluetooth_search]

Экран (\ref{bluetooth_search}) поиска пользователей c использованием bluetooth.

\subsection{Функциональные требования}

\paragraph{} %1

Создание аккаунта и авторизация происходит на экране авторизации (\ref{auth}).
От пользователя требуется ввести email и пароль затем нажать кнопку-действие.

\illustration[][][0.3]{img/auth_email_password}[auth_email_password]

Введем email и пароль (\ref{auth_email_password}). Затем нажмем кнопку регистрации "Sign up". При условии, что данные введены верно, мы должны попасть на главный экран чатов (\ref{chats_not_empty}), он должен быть пуст.

\paragraph{} %2

Поиск пользователей по интентификатору (email) происходит на экране поиска по email (\ref{search_users}). Введем запрос в поисковую строку вверху экрана.

\illustration[][][0.3]{img/search_user_result}[search_user_result]

В списке появится результат поиска (\ref{search_user_result}). Нажав на имя пользователя мы попадем на экран сообщений ии сможем написать ему.

\paragraph{} %3

При нажатии на имя пользователя из списка результатов поиска (\ref{search_user_result}), или при нажатии на чат на экране списков чатов (\ref{chats_not_empty}) откроется экран списка сообщений, в нижней части которого расположено поле для ввода и отправки сообщений.

\illustration[][][0.3]{img/messaging_type}[messaging_type]

Пример сообщения введенного пользователем (\ref{messaging_type}).

\illustration[][][0.3]{img/messaging_sent}[messaging_sent]

После нажатия на кнопку отправить (\ref{messaging_sent}), сообщение отправится собеседнику.

\paragraph{} %5

Возможность просматривать сообщения обеспечивается с помощью экрана чата (\ref{messaging_type}).

\paragraph{} %6

Функция начать переписку с помощью Bluetooth реализована с помощью экрана поиска по блютуз (\ref{bluetooth_search}). После того как пользователь заходит в этот экран начинается сканирование пользователей вокруг. Формируется список пользователей найденных по близости.

\illustration[][][0.3]{img/messaging_start}[messaging_start]

При нажатии на имя пользователя можно начать с ними переписку (\ref{messaging_start}).

\paragraph{} %7

Возможность удаления чата с другим пользователем реализована по средствам контексного меню. Перейдем на экран списка чатов (\ref{chats_not_empty}). При долгом нажатии на ячеку чата откроестя контекстное меню, в опции которого можно выбрать "Delete current chat".

\paragraph{} %8

Для удаления собственного аккаунта перейдем в меню настройки (\ref{settings}), прии нажатии на кнопку Delete account, аккаунт пользователя будет удален, он выйдет из аккаунта, и все его данные будут удалены.

\paragraph{} %9

Возможность смены пароля реализована на экране авторизации (\ref{auth}).
Пользователь должен ввести email от аккаунта, пароль которого он хочет востановить, а затем нажать на кнопку "Forgot password". После чего на указанную электронную почту прийдут дальнейшие инструкции.

\subsection{Проверка выполнения требований к надежности}

\subsection{Обработка и показ ошибок}

На экране авторизации (\ref{auth}), при вводе некорректных данных email или пароля появится диалоговое окно с ошибкой. 

\illustration[][][0.3]{img/auth_error}[auth_error]

Таким образом осуществляется оповещение пользователей о направильном вводе данных.  

\bibliography{espd,library}

\end{document}

\documentclass[testmethods]{espd}
\usepackage[russian]{babel}

\bibliographystyle{gost2008}

\managerrank{Научный руководитель,\\доцент департамента\\программной инженерии\\факультета компьютерных наук,\\канд. техн. наук}
\manager{С.Л. Макаров}

\authorrank{студент группы БПИ183}
\author{М.И. Сердюков}

\title{Мессенджер с поиском по Bluetooth\\(клиентская часть)}
\code{02.07}

\city{Москва}
\year{2021}

\begin{document}

\annotation

Программа и методика испытаний -- это документ, содержащий набор требований к программному продукту и программной документации, а также содержащий методику испытаний программного продукта.

Настоящая Программа и методика испытаний продукта <<Мессенджер с поиском по Bluetooth>> содержит следующие разделы: <<Объект испытаний>>, <<Цель испытаний>>, <<Требования к программе>>, <<Требования к программной документации>>, <<Средства и порядок испытаний>>, <<Методы испытаний>>.

В разделе <<Объект испытаний>> указано наименование программы, область применения и обозначение программы.

В разделе <<Цель испытаний>> указана цель испытаний программного продукта.

В разделе <<Требования к программе>> указаны требования по функционалу и надежности к разрабатываемому продукту.

В разделе <<Требования к программной документации>> указан состав программной документации.

В разделе <<Средства и порядок испытаний>> указаны средства, с помощью которых проводятся испытания и указан порядок испытаний.

В разделе <<Методы испытаний>> содержится описание действий, требуемых для испытания программного продукта и приведен результат испытаний.

Настоящая Программа и методика испытаний удовлетворяет требованиям ГОСТ 19.301-78~\cite{espd301}.

\tableofcontents

\section{Объект испытаний}
\subsection{Наименование программы}
\paragraph{Наименование программы на русском языке}
Мессенджер с поиском по Bluetooth.
\paragraph{Наименование программы на английском языке}
Messenger with Bluetooth Search.

\subsection{Область применения}
Программа предназначена для контроля доступа пользователей, обеспечения возможности передачи текстовых сообщений между пользователями, а также для обеспечения возможности обмена сообщениями между двумя находящимися близко пользователями, с использованием сигнатур Bluetooth.

Программа применяется как серверная часть общего продукта, вместе с клиентом -- мобильным приложением.

\subsection{Обозначение испытуемой программы}
Краткое обозначение испытуемой программы -- <<closecom>>.

\section{Цель испытаний}
\subsection{Цель проведения испытаний}
Целью испытаний является проверка заявленных в Техническом задании требований к функциональности и надежности программного комплекса. Проведение испытаний обосновано в Техническом задании.

\section{Требования к программе}
\subsection{Требования к функциональным характеристикам}
\paragraph{API интерфейс}
Требования относятся к REST API интерфейсу сервера для клиентского приложения:

\begin{enumerate}
\item Создание аккаунта с заданием почты и пароля;
\item Задание и изменение данных аккаунта;
\item Удаление аккаунта;
\item Получение списка текущих диалогов;
\item Получение сообщений указанного диалога;
\item Отправка сообщения указанному пользователю;
\item Удаление указанного сообщения или диалога;
\item Получение списка контактов, которые находятся рядом с пользователем.
\end{enumerate}

\paragraph{Поддержка}

\begin{enumerate}
\item Программа должна составлять логи для анализа ее работы;
\item Должна использоваться виртуализация для обеспечения надежности работы;
\item Исходный код должен быть расположен в git-репозитории;
\item Секретные данные (как данные для доступа к почтовому сервису) должны быть вынесены в отдельный файл секретов.
\end{enumerate}

\subsection{Требования к надежности}
Приложение должно проверять входные данные для обеспечения корректной работы. Ошибки должны обрабатываться на уровне веб-фреймворка и не влиять на работу сервера в целом.

\section{Требования к программной документации}
\subsection{Состав программной документации}\label{subsection:documentation}
<<Мессенджер с поиском по Bluetooth>>. Техническое задание (ГОСТ 19.201-78~\cite{espd201})

<<Мессенджер с поиском по Bluetooth>>. Программа и методика испытаний (ГОСТ 19.301-78~\cite{espd301})

<<Мессенджер с поиском по Bluetooth>>. Пояснительная записка (ГОСТ 19.404-79~\cite{espd404})

<<Мессенджер с поиском по Bluetooth>>. Руководство рограммиста (ГОСТ 19.504-79~\cite{espd505})

<<Мессенджер с поиском по Bluetooth>>. Текст программы (ГОСТ 19.401-78~\cite{espd401})

\subsection{Специальные требования к программной документации}
Документы к программе должны быть выполненны в соответствии с ГОСТ 19.106-78~\cite{espd106} и ГОСТами к каждому виду документа (см. п.~\ref{subsection:documentation}).

Пояснительная записка должна быть загружена в систему <<Антиплагиат>> через LMS НИУ ВШЭ.

Документация и программа сдаются в электронном виде в формате .pdf или .docx в архиве формата .zip или .rar.

За три дня до защиты комиссии все материалы курсового проекта:
\begin{enumerate}
\item Техническая документация;
\item Программный проект;
\item Исполняемый файл;
\item Отзыв руководителя;
\item Лист Антиплагиата.
\end{enumerate}
должны быть загружены одним или несколькими архивами в проект дисциплины <<Курсовой проект 2020-2021>> в личном кабинете информационной образовательной среде LMS (Learning Management System) НИУ ВШЭ.

\section{Средства и порядок испытаний}
\subsection{Технические и программные средства, используемые во время испытаний}
Во время испытаний в качестве сервера используется VM instance Google Cloud Platform уровня f1-micro, который удовлетворяет минимальным требованиям к серверу. Произведены все операции, указанные в Руководстве программиста.

\subsection{Порядок проведения испытаний}
\begin{enumerate}
\item Проверка выполнения требований к интерфейсу;
\item Проверка выполнения требований к поддержке.
\end{enumerate}

\section{Методы испытаний}
\subsection{Описание используемых методов испытаний для проверки программной документации}
Документация проверяется на соответствие составу (см. п.~\ref{subsection:documentation}). Также должно быть соответствие стандартам ГОСТ 19 ЕСПД, указанным в этом пункте.

\subsection{Описание используемых методов испытаний для проверки требований к интерфейсу}
Для удобства проведения испытаний запросы будем делать из сервера, т.е. по аресу localhost.

\paragraph{} %1
Ввод:

\begin{verbatim}
curl -X POST 127.0.0.1:8000/account/create \
-d '{"email": "party_50@mail.ru", "password": "12345678"}' \
-H  "Content-Type: application/json"
\end{verbatim}

Вывод:

\begin{verbatim}
{"token":"TOKEN"}
\end{verbatim}

Как видим, при соответствующем запросе создается пользователь. При этом на указанный почтовый ящик приходит сообщение о подтверждении аккаунта, а в теле ответа передается токен доступа. В следующих командах мы будем использовать этот токен.

\paragraph{} %2
Ввод:

\begin{verbatim}
curl -X POST '127.0.0.1:8000/account/set_about?token=TOKEN' \
-d '{"name": "Denis Moldavsky"}' \
-H  "Content-Type: application/json"
\end{verbatim}

Вывод:

\begin{verbatim}
{"message":"ok"}
\end{verbatim}

Данные о пользователе были заданы в системе, мы их увидим при следующих командах.

\paragraph{} %3
Для этого пункта был создан второй аккаунт

Ввод:

\begin{verbatim}
curl -X POST '127.0.0.1:8000/account/delete?token=TOKEN'
\end{verbatim}

Вывод:

\begin{verbatim}
{"message":"ok"}
\end{verbatim}

Авторизация для данного аккаунта больше не работает, следовательно, данное требование выполнено.

\paragraph{} %4
Сначала создадим второго пользователя и отправим сообщение, инициируя диалог, используя обработчик /start\_dialog (при этом мы получили id чата - 1). Теперь получим информацию о всех диалогах.

Ввод:

\begin{verbatim}
curl 127.0.0.1:8000/messenger/chats?token=TOKEN
\end{verbatim}

Вывод:

\begin{verbatim}
{
  "chats": [
    {
      "chat_id": 1,
      "latest_message": {
        "chat_id": 1,
        "id": 1,
        "text": "Hello Misha!1111 :D",
        "time": "2021-04-11T12:18:01.386Z",
        "user_id": 1
      },
      "name": "misha@mailthatnotexists.ru",
      "senders": [
        {
          "about": {
            "name": "Denis Moldavsky"
          },
          "confirmed": 0,
          "email": "party_50@mail.ru",
          "id": 1
        }
      ],
      "type": "private",
      "users": [
        {
          "about": {
            "name": "Denis Moldavsky"
          },
          "confirmed": 0,
          "email": "party_50@mail.ru",
          "id": 1
        },
        {
          "about": {
            "name": null
          },
          "confirmed": 0,
          "email": "misha@mailthatnotexists.ru",
          "id": 3
        }
      ]
    }
  ]
}
\end{verbatim}

 Чат создается и возвращается.

\paragraph{} %5
Ввод:

\begin{verbatim}
curl '127.0.0.1:8000/messenger/chat_history?token=TOKEN&chat_id=1'
\end{verbatim}

Вывод:

\begin{verbatim}
{
  "messages": [
    {
      "chat_id": 1,
      "id": 1,
      "text": "Hello Misha!1111 :D",
      "time": "2021-04-11T12:18:01.386Z",
      "user_id": 1
    }
  ],
  "name": "misha@mailthatnotexists.ru",
  "senders": [
    {
      "about": {
        "name": "Denis Moldavsky"
      },
      "confirmed": 0,
      "email": "party_50@mail.ru",
      "id": 1
    }
  ],
  "type": "private",
  "users": [
    {
      "about": {
        "name": "Denis Moldavsky"
      },
      "confirmed": 0,
      "email": "party_50@mail.ru",
      "id": 1
    },
    {
      "about": {
        "name": null
      },
      "confirmed": 0,
      "email": "misha@mailthatnotexists.ru",
      "id": 3
    }
  ]
}
\end{verbatim}

Таким образом мы получаем список всех сообщений в чате.

\paragraph{} %6
Ввод:

\begin{verbatim}
curl -X POST '127.0.0.1:8000/messenger/send_message?token=TOKEN&chat_id=1' \
-d '{"text": "Hello Misha!1111 :D", "time": "2021-04-11T12:18:01.386Z"}' \
-H  "Content-Type: application/json"
\end{verbatim}

Вывод:

\begin{verbatim}
{"message":"ok"}
\end{verbatim}

Используя команду /chat\_history, которая использовалась выше, определяем, что сообщение действительно было отправлено.

\paragraph{} %7
Ввод:

\begin{verbatim}
curl -X POST '127.0.0.1:8000/messenger/delete_message?token=TOKEN&message_id=2'
\end{verbatim}

Вывод:

\begin{verbatim}
{"message":"ok"}
\end{verbatim}

Используя ту же команду /chat\_history, проверяем, что сообщение действительно было удалено. Следовательно, требование выполнено. При этом команда /delete\_chat работает аналогично, вывод совпадает с выводом /delete\_message.

\paragraph{} %8
Сначала установим для второго пользователя bid (Bluetooth ID) -- подстроку, которая используется на стороне клиента для определения устройства пользователя, которое находится рядом. Сделаем это с помощью команды /set\_bid. Теперь осталось от имени второго пользователя получить информацию о первом, используя этот bid:

Ввод:

\begin{verbatim}
curl '127.0.0.1:8000/bluetooth/user_about?token=TOKEN&bid=test:mac:address'
\end{verbatim}

Вывод:

\begin{verbatim}
{
  "about": {
    "name": null
  },
  "confirmed": 0,
  "email": "misha@mailthatnotexists.ru",
  "id": 3
}
\end{verbatim}

\subsection{Описание используемых методов испытаний для проверки требований к поддержке}
\paragraph{} %1
В ходе проверки требований к интерфейсу были записаны соответствующие сообщение в директорию logs. Соответственно, программа пишет логи и она доступна к анализу.

\paragraph{} %2
В Руководстве программиста было описано, как необходимые Python-зависимости программы устанавливаются с помощью библиотеки virtualenv. Соответственно, требование о виртуализации выполнено.

\paragraph{} %3
Отправка почтовых сообщений пользователям работает, секретные данные (см. Руководство программиста) сохраняются на сервере отдельно от исполняемого кода. Данное требование выполнено.

\subsection{Проверка выполнения требований к надежности}
Для проверки надежности в коде обработчика send\_message была намеренно допущена ошибка (вызов деления целочисленного числа на 0). После запроса к данному обработчику было возвращено сообщение:

\begin{verbatim}
{"error_message": "internal server error"}
\end{verbatim}

с кодом возврата 500. Соответствующие записи появились в access и error логах (см. Руководство программиста). Соответственно, данное требование выполнено.

\bibliography{espd,library}

\end{document}

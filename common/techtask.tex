\documentclass[techtask]{espd}
\usepackage[russian]{babel}
\usepackage{multirow}
\usepackage{array}

\bibliographystyle{gost2008}

\managerrank{Научный руководитель,\\доцент департамента\\программной инженерии\\факультета компьютерных наук,\\канд. техн. наук}
\manager{С.Л. Макаров}

\authorranki{студент группы БПИ183}
\authori{Д.А. Молдавский}

\authorrankii{студент группы БПИ183}
\authorii{М.И. Сердюков}

\title{Мессенджер с поиском по Bluetooth}
\code{04.03}

\city{Москва}
\year{2020}

\begin{document}

\annotation
Техническое задание -- это основной документ, оговаривающий набор требований и порядок создания программного продукта, в соответствии с которым производится разработка программы, ее тестирование и приемка.

Настоящее Техническое задание на разработку <<Мессенджера с поиском по Bluetooth>> содержит следующие разделы: <<Введение>>, <<Основания для разработки>>, <<Назначение разработки>>, <<Требования к программе>>, <<Требования к программной документации>>, <<Технико-экономические показатели>>, <<Стадии и этапы разработки>>, <<Порядок контроля и приемки>>.

В разделе <<Введение>> указаны наименование программы и краткая характеристика области применения разработки.

В разделе <<Основания для разработки>> указан документ, на основании которого ведется разработка, а также наименование темы разработки.

В разделе <<Назначение разработки>> указано функциональное и эксплуатационное назначение программного продукта.

В разделе <<Требования к программе>> указаны требования по функционалу и надежности к разрабатываемому продукту, а также условия, накладываемые на технические и информационные средства, в условиях которых предполагается использование программного продукта.

В разделе <<Требования к программной документации>> указан предварительный состав программной документации.

В разделе <<Технико-экономические показатели>> указана предполагаемая потребность, а также экономические преимущества разработки по сравнению с аналогами.

В разделе <<Стадии и этапы разработки>> содержится информация о стадиях разработки и их содержании, а также сроки разработки и исполнители.

В разделе <<Порядок контроля и приемки>> указаны общие требования к приемке работы.

Настоящее Техническое задание удовлетворяет требованиям ГОСТ 19.201-79~\cite{espd201}.

\tableofcontents

\section{Введение}
\subsection{Наименование программы}
\paragraph{Наименование программы на русском языке}
Мессенджер с поиском по Bluetooth.
\paragraph{Наименование программы на английском языке}
Messenger with Bluetooth Search.

\subsection{Краткая характеристика области применения}
Основной областью применения данной программы является обмен сообщениями (в виде текста и файлов) с функцией быстрого установления контакта (без необходимости переписывать номер телефона, логин или другой идентификатор. Такое приложение может использоваться в случаях, когда обмен информацией лично может быть затруднен, например, при создании чата группой человек, находящихся возле друг друга.

\section{Основания для разработки}
\subsection{Документ, на основании которого ведется разработка}
Разработка ведется на основании приказа декана факультета компьютерных наук И.В. Аржанцева "Об утверждении тем, руководителей курсовых работ студентов образовательной программы «Программная инженерия» факультета компьютерных наук" № 0.0-00/0000-00 от 00.00.0000.

\subsection{Наименование темы разработки}
\paragraph{Наименование темы разработки на русском языке}
Мессенджер с поиском по Bluetooth.

\paragraph{Наименование темы разработки на английском языке}
Messenger with Bluetooth Search.

\section{Назначение разработки}
\subsection{Функциональное назначение программы}
Программа предназначена для передачи сообщений (в виде текста и файлов), между пользователями.

\subsection{Эксплуатационное назначение программы}
Программа предназначена для эксплуатации обычными пользователями, у которых существует потребность в общении через интернет, используя мобильные устройства.

\section{Требования к программе}
\subsection{Требования к функциональным характеристикам}
Программа должна обеспечивать следующий функционал:

\begin{enumerate}
\item Авторизация через электронную почту и пароль (создание аккаунта, регистрация);
\item Поиск пользователей, находящихся рядом по Bluetooth;
\item Возможность передачи сообщений между пользователями (в виде текста и файлов) и их удаление;
\item Возможность редактирования данных пользователя.
\end{enumerate}

\paragraph{Требования к организации входных данных}
Специфицируется документами на разработку клиентской и серверной части отдельно.

\paragraph{Требования к организации выходных данных}
Специфицируется документами на разработку клиентской и серверной части отдельно.

\subsection{Требования к надежности}
Программа должна проверять входные данные и корректно работать при любых сценариях пользователя.

\subsection{Условия эксплуатации}
Специфицируется документами на разработку клиентской и серверной части отдельно.

\subsection{Требования к составу и параметрам технических средств}\label{subsection:requirements}
Специфицируется документами на разработку клиентской и серверной части отдельно.

\subsection{Требования к информационной и программной совместимости}
Специфицируется документами на разработку клиентской и серверной части отдельно.

\subsection{Требования к маркировке и упаковке}
Специфицируется документами на разработку клиентской и серверной части отдельно.

\subsection{Требования к транспортированию и хранению}
Требования к транспортировке и хранению программных документов являются стандартными и должны соответствовать общим требованиям хранения и транспортировки печатной продукции:

\begin{enumerate}
\item В помещении для хранения печатной продукции допустимы температура воздуха от 10 $^\circ$С до 30 $^\circ$С и относительная влажность воздуха от 30\% до 60\%;
\item Документацию хранят и используют на расстоянии не менее 0.5 м от источников тепла и влаги. Не допускается хранение печатной продукции в помещениях, где находятся агрессивные агенты – растворители, спирт, бензин;
\item Не допускается попадание на документацию агрессивных агентов;
\item Транспортировка производится в специальных контейнерах с применением мер по предотвращению деформации документов внутри контейнеров, а также проникновения влаги, вредных газов, пыли, солнечных лучей и образованию конденсата внутри контейнеров;
\item Программные документы, предоставляемые в печатном виде, должны соответствовать общим правилам учета и хранения программных документов, предусмотренных стандартами Единой системы программной документации и соответствовать требованиям ГОСТ 19.602-78~\cite{espd602}.
\end{enumerate}

\section{Требования к программной документации}
\subsection{Предварительный состав программной документации}\label{subsection:documentation}
<<Мессенджер с поиском по Bluetooth>>. Техническое задание (ГОСТ 19.201-78~\cite{espd201})

<<Мессенджер с поиском по Bluetooth>>. Программа и методика испытаний (ГОСТ 19.301-78~\cite{espd301})

<<Мессенджер с поиском по Bluetooth>>. Пояснительная записка (ГОСТ 19.404-79~\cite{espd404})

<<Мессенджер с поиском по Bluetooth>>. Руководство оператора (ГОСТ 19.505-79~\cite{espd505})

<<Мессенджер с поиском по Bluetooth>>. Текст программы (ГОСТ 19.401-78~\cite{espd401})

\subsection{Специальные требования к программной документации}\label{subsection:docspec}
Документы к программе должны быть выполнены в соответствии с ГОСТ 19.106-78~\cite{espd106} и ГОСТами к каждому виду документа (см. п.~\ref{subsection:documentation}).

Пояснительная записка должна быть загружена в систему <<Антиплагиат>> через LMS НИУ ВШЭ.

Документация и программа сдаются в электронном виде в формате .pdf или .docx в архиве формата .zip или .rar.

За один день до защиты комиссии все материалы курсового проекта:
\begin{enumerate}
\item Техническая документация;
\item Программный проект;
\item Исполняемый файл;
\item Отзыв руководителя;
\item Лист Антиплагиата.
\end{enumerate}
должны быть загружены одним или несколькими архивами в проект дисциплины <<Курсовой проект 2020-2021>> в личном кабинете информационной образовательной среде LMS (Learning Management System) НИУ ВШЭ.

\section{Технико-экономические показатели}
\subsection{Предполагаемая потребность}
Предполагается, что программа будет использоваться обычными пользователями, для общения по сети, при помощи мобильного устройства.

\subsection{Экономические преимущества разработки по сравнению с отечественными и зарубежными аналогами}
Отличительной особенностью данного приложения от аналогов (Telegram~\cite{telegram}, WhatsApp~\cite{whatsapp}) является наличие возможности поиска пользователей с помощью Bluetooth.

\section{Стадии и этапы разработки}
\subsection{Необходимые стадии разработки, этапы и содержание работ}

\noindent\begin{tabular}{|>{\raggedright}p{50mm}|>{\raggedright}p{55mm}|>{\raggedright\arraybackslash}p{60mm}|}
\hline
Стадии & Этапы работ и ответственный & Содержание работ \\ \hline
\multirow[t]{8}{=}{1. Техническое задание} & \multirow[t]{2}{=}{Обоснование необходимости разработки программы (Молдавский)} & Постановка исходных материалов \\ \cline{3-3}
& & Сбор исходных материалов \\ \cline{2-3}
& \multirow[t]{3}{=}{Научно-исследовательские работы (Молдавский)} & Определение структуры входных и выходных данных \\ \cline{3-3}
& & Определение требований к техническим средствам \\ \cline{3-3}
& & Обоснование принципиальной возможности решения поставленной задачи \\ \cline{2-3}
& \multirow[t]{3}{=}{Разработка и утверждение технического задания (Молдавский)} & Определение требований к программе \\ \cline{3-3}
& & Определение стадий, этапов и сроков разработки программы и документации на нее \\ \cline{3-3}
& & Согласование и утверждение технического задания \\ \hline
\multirow[t]{5}{=}{2. Рабочий проект} & Разработка серверной части (Молдавский) & Программирование и отладка серверного кода \\ \cline{2-3}
& Разработка клиентской части (Сердюков) & Программирование и отладка кода мобильной программы \\ \cline{2-3}
& Разработка программной документации сервера (Молдавский) & Разработка программных документов в соответствии с требованиями ГОСТ 19.101-77~\cite{espd101} \\ \cline{2-3}
& Разработка программной документации клиента (Сердюков) & Разработка программных документов в соответствии с требованиями ГОСТ 19.101-77~\cite{espd101} \\ \cline{2-3}
& Испытания программы (Сердюков) & Разработка, согласование и утверждение порядка и методики испытаний \\ \hline
\end{tabular}

\noindent\begin{tabular}{|>{\raggedright}p{50mm}|>{\raggedright}p{55mm}|>{\raggedright\arraybackslash}p{60mm}|}
\hline
\multirow[t]{6}{=}{3. Внедрение} & \multirow[t]{4}{=}{Подготовка и передача программы} & Утверждение даты защиты программного продукта \\ \cline{3-3}
& & Подготовка программы и программной документации для презентации и защиты \\ \cline{3-3}
& & Представление разработанного программного продукта руководителю и получение отзыва \\ \cline{3-3}
& & Загрузка Пояснительной записки в систему Антиплагиат через LMS НИУ ВШЭ \\ \cline{3-3}
& & Загрузка материалов курсового проекта (курсовой работы) в LMS, проект дисциплины <<Курсовой проект 2020-2021>> (см. п.~\ref{subsection:docspec}) \\ \cline{3-3}
& & Защита программного продукта (курсового проекта) комиссии \\ \hline
\end{tabular}

\subsection{Сроки разработки и исполнители}
Разработка должна закончиться к 00 мая 2021 года.

\section{Порядок контроля и приемки}
\subsection{Виды испытаний}
Проверка программного продукта, в том числе и на соответствие техническому заданию, осуществляется исполнителем вместе с заказчиком согласно <<Программе и методике испытаний>>, а также пункту~\ref{subsection:docspec}.

\subsection{Общие требования к приемке работы}
Защита выполненного проекта осуществляется комиссии, состоящей из преподавателей департамента программной инженерии, в утвержденные приказом декана ФКН сроки.

\bibliography{espd,library}

\end{document}

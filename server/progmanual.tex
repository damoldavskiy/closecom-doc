\documentclass[progmanual]{espd}
\usepackage[russian]{babel}

\bibliographystyle{gost2008}

\managerrank{Научный руководитель,\\доцент департамента\\программной инженерии\\факультета компьютерных наук,\\канд. техн. наук}
\manager{С.Л. Макаров}

\authorrank{студент группы БПИ183}
\author{Д.А. Молдавский}

\title{Мессенджер с поиском по Bluetooth\\(серверная часть)}
\code{02.07}

\city{Москва}
\year{2021}

\begin{document}

\annotation

Руководство программиста -- это документ, содержащий сведения для обеспечения корректного использования программного продукта программистом, его развертывания и поддержки.

Настоящее Руководство оператора программы <<Мессенджер с поиском по Bluetooth>>  содержит следующие разделы: <<Назначение программы>>, <<Условия выполнения программы>>, <<Выполнение программы>>, <<Входные и выходные данные>>, <<Сообщения>>.

В разделе <<Назначение программы>> указано эксплуатационное назначение и краткая характеристика функциональности программы.

В разделе <<Условия выполнения программы>> указаны требования к аппаратным и программным средствам, используемым программистом.

В разделе <<Выполнение программы>> указана последовательность действий для корректного запуска программы.

В разделе <<Входные и выходные данные>> приведено описание организации используемой входные и выходной информации.

В разделе <<Сообщения>> указаны основные тексты сообщений программы, их структура и последовательность трактовки.

Настоящее Руководство программиста удовлетворяет требованиям ГОСТ 19.504-79~\cite{espd504}.

\tableofcontents

\section{Назначение программы}
Программа предназначена для обеспечения функциональности для клиентского приложения обмена текстовыми сообщениями между пользователями, их авторизации, хранения пользовательской информации и поиска ближайших контактов.

\section{Условия выполнения программы}
\subsection{Требования к аппаратным средствам}
Для работы приложения небходим сервер со следующими характеристиками:

\begin{enumerate}
\item 1 ядро CPU;
\item 0.6 ГБ RAM;
\item 10 Гб дискового пространства;
\item Доступ в интернет.
\end{enumerate}

Используемый тариф f1-micro GCP соответствует данным требованиям.

\subsection{Требования к программным средствам}
На сервере должна быть установлена операционная система Ubuntu 20.04 LTS minimal.

\section{Выполнение программы}
Далее будет описано, как корректно запустить программу на сервере.

\subsection{Подготовка программного окружения}
Необходимо установить нужные программы, которые не входят в дистрибуцию Ubuntu:

\begin{verbatim}
sudo apt install pip3
sudo apt install nginx
sudo apt install vim
\end{verbatim}

Далее устанавливаем библиотеку Python для создания виртуального окружения и веб-сервер:

\begin{verbatim}
pip3 install virtualenv
pip3 install gunicorn
\end{verbatim}

\subsection{Подготовка файловой системы}
В домашней директории пользователя, от которого предполагается запускать программу, нужно создать следующую схему директорий:

\begin{verbatim}
mkdir data
mkdir logs
mkdir runtime

touch logs/access.log
touch logs/error.log
\end{verbatim}

Далее в папке runtime создаем виртуальное окружение, активируем его и установим остальные библиотеки:

\begin{verbatim}
virtualenv env
source env/bin/activate

pip3 install flask
pip3 install sqlite3
\end{verbatim}

\subsection{Подготовка данных}
Создадим в папке data файл secrets.json, со следующей схемой:

\begin{verbatim}
vim secrets.json

{
    "mail_username": "MAIL",
    "mail_password": "PASSWORD",
    "host_address": "HOST"
}
\end{verbatim}

где MAIL -- почтовый ящик Google, с которого предполагается отправлять сообщения для подтверждения операций с аккаунтами, PASSWORD -- пароль от этого почтового ящика, HOST -- адрес текущего сервера (если нет специального домена, это IP адрес).

\subsection{Установка сервера}
В папке runtime установим исходный код программы:

\begin{verbatim}
git clone https://github.com/damoldavskiy/closecom-server
\end{verbatim}

Запустим миграцию для настройки базы данных:

\begin{verbatim}
cd closecom-server
python3 run_migration.py migrations/v01_init.sql
\end{verbatim}

Теперь можно выйти из виртуального окружения:

\begin{verbatim}
deactivate
\end{verbatim}

Создадим в папке runtime файлы для основных операций с сервером:

\begin{enumerate}
\item Запуск (приложение \ref{attachment:run.sh});
\item Остановка (приложение \ref{attachment:stop.sh});
\item Обновление с репозиторием (приложение \ref{attachment:update.sh}).
\end{enumerate}

\subsection{Конфиг nginx}
Осталось заполнить конфигурационный файл nginx (приложение \ref{attachment:nginx.conf}):

\begin{verbatim}
vim /etc/nginx/nginx.conf
\end{verbatim}

\subsection{Запуск}
Осталось только запустить сервер:

\begin{verbatim}
/home/user/runtime/run.sh
\end{verbatim}

В домашней директории должен появиться файл runtime.pid -- это значит, запуск произошел успешно.

\section{Сообщения оператору}
\subsection{Виды сообщений}
Сообщения программы пишутся в логи и могут быть двух основных видов:
\begin{enumerate}
\item Access-логи;
\item Error-логи.
\end{enumerate}

В текущей конфигурации access-логи пишутся в файл logs/access.log дублируют логи доступа nginx и сохраняют информацию о запросах (время, их URL) и ответ сервера (код). Может быть использован для оценки нагрузки и изучении характеристики использования сервера сторонними ресурсами.

Error-логи пишутся в файл logs/error.log и содержат вывод непосредственно программы при работе. Здесь нужно различать INFO-вывод программы (данные при обработке запросов) и INFO-вывод веб-сервера (служебная информация, запуск/остановка и ошибки).

\subsection{Трактовка логов}

INFO-вывод веб-сервера можно определить по наличию слова Gunicorn в сообщении. Особое внимание необходимо обращать на сообщения, содержащие упоминание Python Exception (в access-логе такие запросы будет иметь код возврата 5xx). О таких сообщениях нужно сообщать разработчику (или дежурнуму, если такой имеется).

INFO-вывод приложения можно определить по тому, что они представляют из себя единственную строку-предложение. На каждый успешный запрос пишется одно такое сообщение. Из них можно узнать базовую информацию о происходящем (например, что пользователь с определенным почтовым ящиком зарегистрировался).

\bibliography{espd,library}

\attachment{Справочное}{Файл run.sh}[attachment:run.sh]
\begin{verbatim}
/home/user/runtime/env/bin/gunicorn main:app \
--pythonpath /home/user/runtime/closecom-server/app \
--config /home/user/runtime/closecom-server/app/constants.py
\end{verbatim}

\attachment{Справочное}{Файл stop.sh}[attachment:stop.sh]
\begin{verbatim}
kill $(cat /home/user/runtime.pid)
\end{verbatim}

\attachment{Справочное}{Файл update.sh}[attachment:update.sh]
\begin{verbatim}
/home/user/runtime/stop.sh
cd /home/user/runtime/closecom-server
git reset --hard
git pull
/home/user/runtime/run.sh
\end{verbatim}

\attachment{Справочное}{Файл nginx.conf}[attachment:nginx.conf]
\begin{verbatim}
events {
    worker_connections 1024;
}

http {
    server {
        location / {
            proxy_pass http://localhost:8000/;
        }
    }
}
\end{verbatim}

\end{document}

\documentclass[techtask]{espd}
\usepackage[russian]{babel}
\usepackage{multirow}
\usepackage{array}

\bibliographystyle{gost2008}

\managerrank{Научный руководитель,\\доцент департамента\\программной инженерии\\факультета компьютерных наук,\\канд. техн. наук}
\manager{С.Л. Макаров}

\authorrank{студент группы БПИ183}
\author{Д.А. Молдавский}

\title{Мессенджер с поиском по Bluetooth\\(серверная часть)}
\code{02.07}

\city{Москва}
\year{2021}

\begin{document}

\annotation
Техническое задание -- это основной документ, оговаривающий набор требований и порядок создания программного продукта, в соответствии с которым производится разработка программы, ее тестирование и приемка.

Настоящее Техническое задание на разработку <<Мессенджера с поиском по Bluetooth>> содержит следующие разделы: <<Введение>>, <<Основания для разработки>>, <<Назначение разработки>>, <<Требования к программе>>, <<Требования к программной документации>>, <<Технико-экономические показатели>>, <<Стадии и этапы разработки>>, <<Порядок контроля и приемки>>.

В разделе <<Введение>> указаны нименование программы и краткая характеристика области применения разработки.

В разделе <<Основания для разработки>> указан документ, на основании которого ведется разработка, а также наименование темы разработки.

В разделе <<Назначение разработки>> указано функциональное и эксплуатационное назначение программного продукта.

В разделе <<Требования к программе>> указаны требования по функционалу и надежности к разрабатываемому продукту, а также условия, накладываемые на технические и информационные средства, в условиях которых предполагается использование программного продукта.

В разделе <<Требования к программной документации>> указан предварительный состав программной документации.

В разделе <<Технико-экономические показатели>> указана предполагаемая потребность, а также экономические преимущества разработки по сравнению с аналогами.

В разделе <<Стадии и этапы разработки>> содержится информация о стадиях разработки и их содержании, а также сроки разработки и исполнители.

В разделе <<Порядок контроля и приемки>> указаны общие требования к приемке работы.

Настоящее Техническое задание удовлетворяет требованиям ГОСТ 19.201-79~\cite{espd201}.

\tableofcontents

\section{Введение}
\subsection{Наименование программы}
\paragraph{Наименование программы на русском языке}
Мессенджер с поиском по Bluetooth.
\paragraph{Наименование программы на английском языке}
Messenger with Bluetooth Search.

\subsection{Краткая характеристика области применения}
Основной областью применения данной программы является обмен сообщениями (в виде текста) с функцией быстрого установления контакта (без необходимости переписывать номер телефона, логин или другой идентификатор. Такое приложение может использоваться в случаях, когда обмен информацией лично может быть затруднен, например, при создании чата группой человек, находящихся возле друг друга.

Настоящее Техническое задание описывает серверную часть приложения, которая является необходимой ля корректного функционирования мобильного клиента.

\section{Основания для разработки}
\subsection{Документ, на основании которого ведется разработка}
Разработка ведется на основании приказа декана факультета компьютерных наук И.В. Аржанцева "Об утверждении тем, руководителей курсовых работ студентов образовательной программы «Программная инженерия» факультета компьютерных наук" № 0.0-00/0000-00 от 00.00.0000.

\subsection{Наименование темы разработки}
\paragraph{Наименование темы разработки на русском языке}
Мессенджер с поиском по Bluetooth.

\paragraph{Наименование темы разработки на английском языке}
Messenger with Bluetooth Search.

\section{Назначение разработки}
\subsection{Функциональное назначение программы}
Программа предназначена для контроля доступа пользователей, обеспечения возможности передачи текстовых сообщений между пользователями, а также для обеспечения возможности обмена сообщениями между двумя находящимися близко пользователями, с использованием сигнатур Bluetooth.

\subsection{Эксплуатационное назначение программы}
Программа предназначена для развертывания и эксплуатации на сервере.

\section{Требования к программе}
\subsection{Требования к функциональным характеристикам}
\paragraph{Требования к составу выполняемых функций}
Должны быть следующие возможности, путем использования REST API:

\begin{enumerate}
\item Создание аккаунта с заданием почты и пароля;
\item Задание и изменение данных аккаунта;
\item Удаление аккаунта;
\item Получение списка текущих диалогов;
\item Получение сообщений указанного диалога;
\item Отправка сообщения указанному пользователю;
\item Удаление указанного сообщения или диалога;
\item Получение списка контактов, которые находятся рядом с пользователем.
\end{enumerate}

Также, серверная часть приложения должна обеспечивать следующий функционал поддержки:

\begin{enumerate}
\item Программа должна составлять логи для анализа ее работы;
\item Должна использоваться виртуализация для обеспечения надежности работы;
\item Исходный код должен быть расположен в git-репозитории;
\item Секретные данные (как данные для доступа к почтовому сервису) должны быть вынесены в отдельный файл секретов.
\end{enumerate}

\paragraph{Требования к организации входных данных}
Программа принимает на вход HTTP запросы, которые должны быть сформированы согласно интерфейсу системы.

\paragraph{Требования к организации выходных данных}
На HTTP запросы программа возвращает ответ согласно описанию интерфейса системы, при необходимости дополнительные данные передаются через тело запроса в JSON формате.

\subsection{Требования к надежности}
Программа должна проверять входные данные и корректно возвращать ответ на любой запрос, независимо от его соответствия интерфейсу.

\subsection{Условия эксплуатации}
Программа должна эксплуатироватья в облачной инфрастуктуре Microsoft Azure.

\subsection{Требования к составу и параметрам технических средств}
Необходимо использование тарифного плана B1s Microsoft Azure или других, не уступающих по техническим характеристикам.

\subsection{Требования к информационной и программной совместимости}
Для работы программы должна быть установлена Ubuntu LTS версии, а также все необходимые для работы библиотеки.

\subsection{Требования к маркировке и упаковке}
Программа поставляется в виде архива с исполняемым файлов для целевой платформы.

\subsection{Требования к транспортированию и хранению}
Требования к транспортировке и хранению программных документов являются стандартными и должны соответствовать общим требованиям хранения и транспортировки печатной продукции:

\begin{enumerate}
\item В помещении для хранения печатной продукции допустимы температура воздуха от 10 $^\circ$С до 30 $^\circ$С и относительная влажность воздуха от 30\% до 60\%;
\item Документацию хранят и используют на расстоянии не менее 0.5 м от источников тепла и влаги. Не допускается хранение печатной продукции в помещениях, где находятся агрессивные агенты – растворители, спирт, бензин;
\item Не допускается попадание на документацию агрессивных агентов;
\item Транспортировка производится в специальных контейнерах с применением мер по предотвращению деформации документов внутри контейнеров, а также проникновения влаги, вредных газов, пыли, солнечных лучей и образованию конденсата внутри контейнеров;
\item Программные документы, предоставляемые в печатном виде, должны соответствовать общим правилам учета и хранения программных документов, предусмотренных стандартами Единой системы программной документации и соответствовать требованиям ГОСТ 19.602-78~\cite{espd602}.
\end{enumerate}

\section{Требования к программной документации}
\subsection{Предварительный состав программной документации}\label{subsection:documentation}
<<Мессенджер с поиском по Bluetooth>>. Техническое задание (ГОСТ 19.201-78~\cite{espd201})

<<Мессенджер с поиском по Bluetooth>>. Программа и методика испытаний (ГОСТ 19.301-78~\cite{espd301})

<<Мессенджер с поиском по Bluetooth>>. Пояснительная записка (ГОСТ 19.404-79~\cite{espd404})

<<Мессенджер с поиском по Bluetooth>>. Руководство оператора (ГОСТ 19.505-79~\cite{espd505})

<<Мессенджер с поиском по Bluetooth>>. Текст программы (ГОСТ 19.401-78~\cite{espd401})

\subsection{Специальные требования к программной документации}\label{subsection:docspec}
Документы к программе должны быть выполненны в соответствии с ГОСТ 19.106-78~\cite{espd106} и ГОСТами к каждому виду документа (см. п.~\ref{subsection:documentation}).

Пояснительная записка должна быть загружена в систему <<Антиплагиат>> через LMS НИУ ВШЭ.

Документация и программа сдаются в электронном виде в формате .pdf или .docx в архиве формата .zip или .rar.

За один день до защиты комиссии все материалы курсового проекта:
\begin{enumerate}
\item Техническая документация;
\item Программный проект;
\item Исполняемый файл;
\item Отзыв руководителя;
\item Лист Антиплагиата.
\end{enumerate}
должны быть загружены одним или несколькими архивами в проект дисциплины <<Курсовой проект 2020-2021>> в личном кабинете информационной образовательной среде LMS (Learning Management System) НИУ ВШЭ.

\section{Технико-экономические показатели}
\subsection{Предполагаемая потребность}
Предполагается, что программа будет использована как функциональная часть клиентского приложения.

\subsection{Экономические преимущества разработки по сравнению с отечественными и зарубежными аналогами}
Достоинства системы по сравнению с аналогами указаны Техническом задании для клиентской части.

\section{Стадии и этапы разработки}
\subsection{Необходимые стадии разработки, этапы и содержание работ}

\noindent\begin{tabular}{|>{\raggedright}p{50mm}|>{\raggedright}p{55mm}|>{\raggedright\arraybackslash}p{60mm}|}
\hline
Стадии & Этапы работ & Содержание работ \\ \hline
\multirow[t]{8}{=}{1. Техническое задание} & \multirow[t]{2}{=}{Обоснование необходимости разработки программы} & Постановка исходных материалов \\ \cline{3-3}
& & Сбор исходных материалов \\ \cline{2-3}
& \multirow[t]{3}{=}{Научно-исследовательские работы} & Определение структуры входных и выходных данных \\ \cline{3-3}
& & Определение требований к техническим средствам \\ \cline{3-3}
& & Обоснование принципиальной возможности решения поставленной задачи \\ \cline{2-3}
& \multirow[t]{3}{=}{Разработка и утверждение технического задания} & Определение требований к программе \\ \cline{3-3}
& & Определение стадий, этапов и сроков разработки программы и документации на нее \\ \cline{3-3}
& & Согласование и утверждение технического задания \\ \hline
\multirow[t]{3}{=}{2. Рабочий проект} & Разработка программы & Программирование и отладка программы \\ \cline{2-3}
& Разработка программной документации & Разработка программных документов в соответствии с требованиями ГОСТ 19.101-77~\cite{espd101} \\ \cline{2-3}
& Испытания программы & Разработка, согласование и утверждение порядка и методики испытаний \\ \hline
\multirow[t]{4}{=}{3. Внедрение} & \multirow[t]{4}{=}{Подготовка и передача программы} & Утверждение даты защиты программного продукта \\ \cline{3-3}
& & Подготовка программы и программной документации для презентации и защиты \\ \cline{3-3}
& & Представление разработанного программного продукта руководителю и получение отзыва \\ \hline
\end{tabular}

\noindent\begin{tabular}{|>{\raggedright}p{50mm}|>{\raggedright}p{55mm}|>{\raggedright\arraybackslash}p{60mm}|}
\hline
\multirow[t]{4}{=}{} & \multirow[t]{4}{=}{} & Загрузка Пояснительной записки в систему Антиплагиат через LMS НИУ ВШЭ \\ \cline{3-3}
& & Загрузка материалов курсового проекта (курсовой работы) в LMS, проект дисциплины <<Курсовой проект 2019-2020>> (см. п.~\ref{subsection:docspec}) \\ \cline{3-3}
& & Защита программного продукта (курсового проекта) комиссии \\ \hline
\end{tabular}

\subsection{Сроки разработки и исполнители}
Разработка должна закончиться к 00 мая 2021 года.

\section{Порядок контроля и приемки}
\subsection{Виды испытаний}
Проверка программного продукта, в том числе и на соответствие техническому заданию, осуществляется исполнителем вместе с заказчиком согласно <<Программе и методике испытаний>>, а также пункту~\ref{subsection:docspec}.

\subsection{Общие требования к приемке работы}
Защита выполненного проекта осуществляется комиссии, состоящей из преподавателей департамента программной инженерии, в утвержденные приказом декана ФКН сроки.

\bibliography{espd,library}

\end{document}


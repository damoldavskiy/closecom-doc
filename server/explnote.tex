\documentclass[explnote]{espd}
\usepackage[russian]{babel}
\usepackage{amsmath}

\bibliographystyle{gost2008}

\managerrank{Научный руководитель,\\доцент департамента\\программной инженерии\\факультета компьютерных наук,\\канд. техн. наук}
\manager{С.Л. Макаров}

\authorrank{студент группы БПИ183}
\author{Д.А. Молдавский}

\title{Программная среда для записи\\математических лекций}
\code{02.07}

\city{Москва}
\year{2021}

\begin{document}

\annotation
Пояснительная записка -- это документ, в котором описаны используемые алгоритмы, принципы функционирования и другая информация, касающаяся работы программы.

Настоящая Пояснительная записка содержит следующие разделы: <<Введение>>, <<Назначение и область применения>>, <<Технические характеристики>>, <<Ожидаемые технико-экономические показатели>>.

В разделе <<Введение>> указано наименование программы и приведены основания для разработки.

В разделе <<Назначение и область применения>> приведено назначение программы и описана характеристика области ее применения.

В разделе <<Технические характеристики>> указана постановка задачи на разработку, используемые в программе алгоритмы, принципы функционирования и методы организации входных и выходных данных.

В разделе <<Ожидаемые технико-экономические показатели>> указана предполагаемая потребность и преимущества разработки над аналогами.

Настоящая Пояснительная записка удовлетворяет требованиям ГОСТ 19.404-79~\cite{espd404}.

\tableofcontents

\section{Введение}
\subsection{Наименование программы}
\paragraph{Наименование программы на русском языке}
Мессенджер с поиском по Bluetooth.
\paragraph{Наименование программы на английском языке}
Messenger with Bluetooth Search.

\subsection{Условное обозначение разработки}
Условное обозначение разработки -- <<closecom>>.

\subsection{Основание для разработки}
Основанием для разработки является учебный план подготовки бакалавров по направлению 09.03.04 <<Программная инженерия>> и утвержденная академическим руководителем тема курсового проекта.

\section{Назначение и область применения}
\subsection{Область применения программы}
Программа применяется в контексте приложения для смартфонов (клиентского приложения). В числе прочих, программная система выполняет функции, которые требуют наличия сервера (хранение и отправка сообщений, данных пользователей, управление данными), что обосновывает необходимость данного приложения.

Программа устанавливается на сервере, который удовлетворяет требованиям (указаны в Техническом задании). Все необходимые функции работы выполняются с помощью HTTP запросов, которые передаются на вход веб-фреймворку и программе.

\subsection{Назначение разработки}
\paragraph{Функциональное назначение программы}
Данная разработка выполняет функции хранения данных о пользователях, чатах, сообщениях, Bluetooth идентификаторов, аутентификации пользователей, долгосрочного хранения этих данных, управлением чатами и сообщениями, а также отправки сообщений на почтовые ящики (для подтверждения различных действий пользователей).

\paragraph{Эксплуатационное назначение программы}
Программа предназначена для эксплуатации программистами или дежурными, которые занимаются поддержкой данного сервера в рамках развития проекта <<closecom>>.

\section{Требования к программе}
\subsection{Требования к функциональным характеристикам}
\paragraph{Требования к составу выполняемых функций}
Программа должна предоставлять следующие возможности, путем использования REST API:

\begin{enumerate}
\item Создание аккаунта с заданием почты и пароля;
\item Задание и изменение данных аккаунта;
\item Удаление аккаунта;
\item Получение списка текущих диалогов;
\item Получение сообщений указанного диалога;
\item Отправка сообщения указанному пользователю;
\item Удаление указанного сообщения или диалога;
\item Получение списка контактов, которые находятся рядом с пользователем.
\end{enumerate}

Также, серверная часть приложения должна обеспечивать следующий функционал поддержки:

\begin{enumerate}
\item Программа должна составлять логи для анализа ее работы;
\item Должна использоваться виртуализация для обеспечения надежности работы;
\item Исходный код должен быть расположен в git-репозитории;
\item Секретные данные (как данные для доступа к почтовому сервису) должны быть вынесены в отдельный файл секретов.
\end{enumerate}

\section{Технические характеристики}
\subsection{Постановка задачи на разработку программы}
Разрабатываемая программа должна обеспечивать работу приложения-клиента, поддерживая все необходимые для его работы обработчики HTTP запросов и соответствующий внутренний функционал. Подробное описание требований содержится в Техническом задании.

\subsection{Описание используеумых алгоритмов}
\paragraph{Устройство сервера}
\subparagraph{Директория runtime}
В папке runtime расположено виртуальное окружение Python, репозиторий и командные файлы для совершения типовых операций с сервером.

Виртуальное окружение предназначено для изоляции зависимостей сервера от пользовательских настроек и библиотек. Это позволяет на этом же физическом сервере производить различные операции в Python не затрагивая непосредственно работающую программу. Данное решение изоляции более предпочтительное, чем использование инструментов виртуализации (Docker, виртуальные машины), так как программа разрабатывается в условиях использования очень ограниченных технических ресурсов.

Репозиторий -- собственно, исполняемый код программы. Является локальной копией удаленного репозитория, что позволяет упростить выпуск новых версий. Таким образом, для выпуска новой версии сервера достаточно обновить удаленный сервер и загрузить ее с помощью встроенных в git инструментов.

Командные файлы -- набор исполняемых файлов bash, которые содержат типовые операции управления веб-сервером gunicorn. Они позволяют быстро запустить, остановить и обновить сервер .

\subparagraph{Директория logs}
В папке logs содержатся логи, которые непрерывно пишутся сервером. Они предзназначены для анализа деятельности сервера, в том числе ошибок. Более детальное их описание содержится в пункте \label{paragraph:output}.

\subparagraph{Директория data}
В папке data содержатся данные сервера, отделенные от репозитория для обеспечения безопасности этих данных (независимости от удаленного репозитория), а также для их сохранности между обновлениями сервера. Здесь используется файл database.db (файл SQLite базы данных) и secrets.json.

secrets.json -- это файл секретов. Секреты -- это данные, которые технически относятся к серверу и не меняются, но не включаются в часть репозитория, так как они должны быть от него изолированы (по причине безопасности). Тут хранятся данные для доступа к почтовому ящику, от которого отправляются e-mail сообщения, а также адрес сервера (который не зависит от кода, но зависит от используемого физического сервера).

\subparagraph{Файл runtime.pid}
Файл runtime.pid содержит id мастер-процесса сервера. Используется для удобного завершения работы программы.

\paragraph{Миграции}
Для организации базы данных используется подход миграций. Это значит, что новые изменения в таблице записываются в новый файл в директории migrations, после чего они выполняются из-под виртуального Python окружения инструментом run_migration. Данный подход позволяет обновлять сервер, не обнуляя данные (и, как следствие, пропадает необходимость в организации специальных транспортов этих данных).

При разработке был использован единственный файл миграций -- v01_init.sql, но после запуска сервера для использования пользователями необходимо будет новые изменения оформлять в виде новых миграций.

\paragraph{Используемые решения в рамках обработчиков}


\paragraph{Авторизация пользователей}

\subsection{Описание и обоснование выбора метода организации входных и выходных данных}
\paragraph{Организация входных данных}
Выходные данные для программы -- это внешние HTTP запросы, которые приложение-клиент посылает на сервер, который, в свою очередь, переадресует их непосредственно обработчику запросов веб-сервера с помощью nginx. Вид запросов согласуется с разработкой мобильного приложения.

\paragraph{Организация выходных данных}\label{paragraph:output}
Выходные данные программы:

\begin{enumerate}
\item HTTP-ответы сервера;
\item логи.
\end{enumerate}

Ответы сервера согласуются с мобильным приложением в рамках интеграции (как и входные данные).

Логи пишутся в директорию logs сервера, они бывают двух видов:

\begin{enumerate}
\item Access-логи;
\item Error-логи.
\end{enumerate}

В текущей конфигурации access-логи пишутся в файл logs/access.log дублируют логи доступа nginx и сохраняют информацию о запросах (время, их URL) и ответ сервера (код). Может быть использован для оценки нагрузки и изучении характеристики использования сервера сторонними ресурсами.

Error-логи пишутся в файл logs/error.log и содержат вывод непосредственно программы при работе. Здесь нужно различать INFO-вывод программы (данные при обработке запросов) и INFO-вывод веб-сервера (служебная информация, запуск/остановка и ошибки).

Более подробное описание логов содержится в Руководстве программиста.

\subsection{Описание и обоснование выбора состава технических средств}
Для обеспечения функционирования программы необходим сервер со следующими характеристиками:

\begin{enumerate}
\item 1 ядро CPU;
\item 0.6 ГБ RAM;
\item 10 Гб дискового пространства;
\item Доступ в интернет.
\end{enumerate}

Такие требования обусловлены минимальными характеристиками сервера, на котором производились разработка и тестирование программы.

\subsection{Описание и обоснование выбора программных средств}
Для работы программы используется Ubuntu 20.04 LTS minimal. Такой выбор объясняется бесплатностью, высокой надежностью и малой потребностью в тезнических средствах данной системы.

\section{Ожидаемые технико-экономические показатели}
Экономическая оцена программы приведена в Пояснительной записке для программной системы. Рассмотрение технико-экономических показателей только в разрезе сервера не имеет ценности.

\bibliography{espd,library}

\begin{terms}
\term{Логи}{данные, которые сервер пишет локально при запросах. Служат для отладки и анализа работы сервера}
\term{Приложение-клиент (мобильное приложение)}{приложение, с которого запросы поступают на сервер}
\term{Переадресация (проксирование) запросов}{переадресация HTTP запросов, которая выполняется nginx. Служит для контроля конечных точек запросов}
\end{terms}

\end{document}


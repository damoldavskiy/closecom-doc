\documentclass[listing]{espd}
\usepackage[russian]{babel}

\bibliographystyle{gost2008}

\managerrank{Научный руководитель,\\доцент департамента\\программной инженерии\\факультета компьютерных наук,\\канд. техн. наук}
\manager{С.Л. Макаров}

\authorrank{студент группы БПИ183}
\author{Д.А. Молдавский}

\title{Мессенджер с поиском по Bluetooth\\(серверная часть)}
\code{02.07}

\city{Москва}
\year{2021}

\begin{document}

\annotation

Текст программы -- это документ, содержащий исходный код программы. Каждому файлу соответствует его подраздел. Ввиду того, что в исходном коде встречаются символы Unicode, данные символы в приведенном ниже коде заменены на обозначения через обратную косую черту. В данном документе содержатся следующие разделы: <<Исходный сервера>>, <<Исходный код миграций>>.

В разделе <<Исходный код сервера>> приведен код непосредственно веб-сервера, основной части данной программы.

В разделе <<Исходный код миграций>> приведен код SQL-миграций для создания базы данных.

Настоящий Текст программы удовлетворяет требованиям ГОСТ 19.401-78~\cite{espd401}.

\tableofcontents
\section{Исходный код сервера}

\subsection{main.py}
\begin{verbatim}
import logging
from flask import Flask, g
from flask_mail import Mail
from constants import ERROR_LOG_PATH, MAIL_SERVER, MAIL_PORT
from controllers import account, bluetooth, messenger
from secrets import secrets
from util import error
app = Flask(__name__)
app.register_blueprint(account.mod)
app.register_blueprint(bluetooth.mod)
app.register_blueprint(messenger.mod)
app.config.update(dict(
    MAIL_SERVER = MAIL_SERVER,
    MAIL_PORT = MAIL_PORT,
    MAIL_USE_TLS = False,
    MAIL_USE_SSL = True,
    MAIL_USERNAME = secrets['mail_username'],
    MAIL_PASSWORD = secrets['mail_password']
))
mail = Mail(app)
logging.basicConfig(filename=ERROR_LOG_PATH,
                    level=logging.INFO,
                    format='[%(asctime)s] [%(process)d] [%(levelname)s
] %(message)s',
                    datefmt='%F %X %z')
@app.errorhandler(500)
def internal_error(exception):
    return error('internal server error', 500)
@app.teardown_appcontext
def close_connection(exception):
    db = getattr(g, '_database', None)
    if db is not None:
        db.close()
\end{verbatim}

\subsection{base.py}
\begin{verbatim}
from models.account import User, Token
from models.messenger import Chat, Message
from util import get_db, get_token
def columns(names, table):
    return ', '.join([f'{table}.{column}' for column in names.split()]
)
USER_COLUMNS = columns('id email password confirmed name bid', 'user')
TOKEN_COLUMNS = columns('id type token user_id', 'token')
CHAT_COLUMNS = columns('id type name', 'chat')
MEMBERSHIP_COLUMNS = columns('user_id chat_id', 'membership')
MESSAGE_COLUMNS = columns('id chat_id user_id time text', 'message')
def get_user_by_id(user_id):
    db = get_db()
    cursor = db.cursor()
    cursor.execute(f'SELECT {USER_COLUMNS} FROM user WHERE id=?', (use
r_id,))
    row = cursor.fetchone()
    if row == None:
        return None
    return User(row)
def get_user_by_email(email):
    db = get_db()
    cursor = db.cursor()
    cursor.execute(f'SELECT {USER_COLUMNS} FROM user WHERE email=?', (
email,))
    row = cursor.fetchone()
    if row == None:
        return None
    return User(row)
def get_user_by_token(token, token_type):
    db = get_db()
    cursor = db.cursor()
    cursor.execute(f'SELECT {TOKEN_COLUMNS} FROM token WHERE token=? A
ND type=?', (token, token_type))
    row = cursor.fetchone()
    if row == None:
        return None
    token = Token(row)
    cursor.execute(f'SELECT {USER_COLUMNS} FROM user WHERE id=?', (tok
en.user_id,))
    row = cursor.fetchone()
    return User(row)
def get_user_by_bid(bid):
    db = get_db()
    cursor = db.cursor()
    cursor.execute(f'SELECT {USER_COLUMNS} FROM user WHERE bid=?', (bi
d,))
    row = cursor.fetchone()
    if row == None:
        return None
    return User(row)
def get_chat_by_id(chat_id):
    db = get_db()
    cursor = db.cursor()
    cursor.execute(f'SELECT {CHAT_COLUMNS} FROM chat WHERE id=?', (cha
t_id,))
    row = cursor.fetchone()
    if row == None:
        return None
    return Chat(row)
def get_message_by_id(message_id):
    db = get_db()
    cursor = db.cursor()
    cursor.execute(f'SELECT {MESSAGE_COLUMNS} FROM message WHERE id=?'
, (message_id,))
    row = cursor.fetchone()
    if row == None:
        return None
    return Message(row)
def create_user(account):
    db = get_db()
    cursor = db.cursor()
    cursor.execute('INSERT INTO user (email, password) VALUES (?, ?)',
 (account.email, account.password))
    cursor.execute(f'SELECT {USER_COLUMNS} FROM user WHERE id=?', (cur
sor.lastrowid,))
    return User(cursor.fetchone())
def delete_user(user):
    db = get_db()
    cursor = db.cursor()
    cursor.execute('DELETE FROM user WHERE email=?', (user.email,))
def create_token(user, token_type):
    db = get_db()
    cursor = db.cursor()
    token = get_token()
    cursor.execute('INSERT INTO token (token, type, user_id) VALUES (?
, ?, ?)', (token, token_type, user.id))
    return token
def set_user_password(user, password):
    db = get_db()
    cursor = db.cursor()
    cursor.execute('UPDATE user SET password=? WHERE id=?', (password,
 user.id))
def set_user_confirmed(user, confirmed):
    db = get_db()
    cursor = db.cursor()
    cursor.execute('UPDATE user SET confirmed=? WHERE id=?', (confirme
d, user.id))
def set_user_about(user, about):
    db = get_db()
    cursor = db.cursor()
    cursor.execute('UPDATE user SET name=? WHERE id=?', (about.name, u
ser.id))
def set_user_bid(user, bid):
    db = get_db()
    cursor = db.cursor()
    cursor.execute('UPDATE user SET bid=? WHERE id=?', (bid, user.id))
def get_chats(user):
    db = get_db()
    cursor = db.cursor()
    cursor.execute('SELECT chat_id FROM membership WHERE user_id=?', (
user.id,))
    rows = cursor.fetchall()
    chats = []
    for row in rows:
        chat_id = row[0]
        cursor.execute(f'SELECT {USER_COLUMNS} FROM user JOIN membersh
ip ON user.id=membership.user_id WHERE chat_id=?', (chat_id,))
        user_rows = cursor.fetchall()
        users = []
        for user_row in user_rows:
            users.append(User(user_row).about())
        cursor.execute(f'SELECT DISTINCT {USER_COLUMNS} FROM user JOIN
 message ON user.id=message.user_id WHERE message.chat_id=?', (chat_id
,))
        sender_rows = cursor.fetchall()
        senders = []
        for sender_row in sender_rows:
            senders.append(User(sender_row).about())
        cursor.execute(f'SELECT {MESSAGE_COLUMNS} FROM message WHERE c
hat_id=? ORDER BY id DESC LIMIT 1', (chat_id,))
        message_row = cursor.fetchone()
        message = Message(message_row).__dict__ if message_row is not 
None else None
        cursor.execute(f'SELECT {CHAT_COLUMNS} FROM chat WHERE id=?', 
(chat_id,))
        chat_row = cursor.fetchone()
        chat = Chat(chat_row)
        chats.append({'chat_id': chat_id, 'type': chat.type, 'name': c
hat.name, 'users': users, 'senders': senders, 'latest_message': messag
e})
    return chats
def create_chat(creator, users, chat_name):
    db = get_db()
    cursor = db.cursor()
    
    cursor.execute('INSERT INTO chat (type, name) VALUES ("public", ?)
', (chat_name,))
    chat_id = cursor.lastrowid
    for user in [creator, *users]:
        cursor.execute('INSERT INTO membership (user_id, chat_id) VALU
ES (?, ?)', (user.id, chat_id))
    return chat_id
def is_user_in_chat(user, chat_id):
     db = get_db()
     cursor = db.cursor()
     cursor.execute(f'SELECT EXISTS(SELECT {MEMBERSHIP_COLUMNS} FROM m
embership WHERE user_id=? AND chat_id=?) AS found', (user.id, chat_id)
)
     row = cursor.fetchone()
     return bool(row[0])
def get_chat_history(chat_id):
    db = get_db()
    cursor = db.cursor()
    cursor.execute(f'SELECT {USER_COLUMNS} FROM user JOIN membership O
N user.id=membership.user_id WHERE chat_id=?', (chat_id,))
    user_rows = cursor.fetchall()
    users = []
    for user_row in user_rows:
        users.append(User(user_row).about())
    cursor.execute(f'SELECT DISTINCT {USER_COLUMNS} FROM user JOIN mes
sage ON user.id=message.user_id WHERE message.chat_id=?', (chat_id,))
    sender_rows = cursor.fetchall()
    senders = []
    for sender_row in sender_rows:
        senders.append(User(sender_row).about())
    cursor.execute(f'SELECT {MESSAGE_COLUMNS} FROM message WHERE chat_
id=? ORDER BY id', (chat_id,))
    message_rows = cursor.fetchall()
    messages = []
    for message_row in message_rows:
        messages.append(Message(message_row).__dict__)
    cursor.execute(f'SELECT {CHAT_COLUMNS} FROM chat WHERE id=?', (cha
t_id,))
    chat_row = cursor.fetchone()
    chat = Chat(chat_row)
    return {'users': users, 'senders': senders, 'type': chat.type, 'na
me': chat.name, 'messages': messages}
def send_message(user, chat_id, message):
    db = get_db()
    cursor = db.cursor()
    cursor.execute('INSERT INTO message (chat_id, user_id, time, text)
 VALUES (?, ?, ?, ?)', (chat_id, user.id, message.time, message.text))
def get_private_chat_id(sender, recipient):
    db = get_db()
    cursor = db.cursor()
    cursor.execute('SELECT EXISTS(SELECT chat.id FROM chat JOIN member
ship AS sender_ms ON chat.id=sender_ms.chat_id JOIN membership AS reci
pient_ms ON chat.id=recipient_ms.chat_id WHERE type="private" AND send
er_ms.user_id=? AND recipient_ms.user_id=?) AS found', (sender.id, rec
ipient.id))
    row = cursor.fetchone()
    if row[0]:
        return 0
    cursor.execute('INSERT INTO chat (type) VALUES ("private")')
    chat_id = cursor.lastrowid
    cursor.execute('INSERT INTO membership (user_id, chat_id) VALUES (
?, ?)', (sender.id, chat_id))
    cursor.execute('INSERT INTO membership (user_id, chat_id) VALUES (
?, ?)', (recipient.id, chat_id))
    return chat_id
def user_search(requester, email):
    db = get_db()
    cursor = db.cursor()
    cursor.execute(f'SELECT {USER_COLUMNS} FROM user WHERE email LIKE 
? LIMIT 10', (f'%{email}%',))
    user_rows = cursor.fetchall()
    users = []
    for user_row in user_rows:
        user = User(user_row)
        if user.email != requester.email:
            users.append(user.about())
    return users
def delete_message(message):
    db = get_db()
    cursor = db.cursor()
    cursor.execute('DELETE FROM message WHERE id=?', (message.id,))
def delete_chat(chat):
    db = get_db()
    cursor = db.cursor()
    cursor.execute('DELETE FROM chat WHERE id=?', (chat.id,))
def delete_membership(user, chat):
    db = get_db()
    cursor = db.cursor()
    cursor.execute('DELETE FROM membership WHERE user_id=? AND chat_id
=?', (user.id, chat.id))
    cursor.execute('SELECT EXISTS(SELECT user_id FROM membership WHERE
 chat_id=?) AS found', (chat.id,))
    row = cursor.fetchone()
    if row[0]:
        return
    delete_chat(chat)
\end{verbatim}

\subsection{secrets.py}
\begin{verbatim}
import json
from constants import SECRETS_PATH
with open(SECRETS_PATH) as secrets_file:
    secrets = json.loads(secrets_file.read())
\end{verbatim}

\subsection{constants.py}
\begin{verbatim}
DATABASE_PATH = '/home/denis/data/database.db'
ACCESS_LOG_PATH = '/home/denis/logs/access.log'
ERROR_LOG_PATH = '/home/denis/logs/error.log'
PIDFILE_PATH = '/home/denis/runtime.pid'
SECRETS_PATH = '/home/denis/data/secrets.json'
MAIL_SERVER = 'smtp.gmail.com'
MAIL_PORT = 465
TOKEN_SIZE = 32
MIN_PASSWORD_LENGTH = 4
# Gunicorn
daemon = True
pidfile = PIDFILE_PATH
errorlog = ERROR_LOG_PATH
accesslog = ACCESS_LOG_PATH
\end{verbatim}

\subsection{util.py}
\begin{verbatim}
import sqlite3
from flask import current_app, g
from flask_mail import Mail, Message
from hashlib import sha256
from os import urandom
from constants import DATABASE_PATH, TOKEN_SIZE
def log_info(message):
    current_app.logger.info(message)
def send_email(recipient, title, text):
    message = Message(title, sender='donotreply@closecom.org', recipie
nts=[recipient])
    message.body = text
    with current_app.app_context():
        mail = Mail()
        mail.send(message)
    
def get_token():
    return urandom(TOKEN_SIZE).hex()
def hash_password(password):
    return sha256(password.encode('utf-8')).hexdigest()
def ok():
    return {'message': 'ok'}
def error(message, code):
    return {'error_message': message}, code
def html_message(title, message):
    return '<html><head><title>' + title + '</title><style>body{color:
#2D3CC8;font-family:sans-serif;'\
           'text-align:center;margin-top:50px;font-size:24;}</style></
head><body>' + message + '</body></html>'
def html_password_change(token):
    return '<html><head><title>Account recovery</title><style>body{col
or:#2D3CC8;font-family:sans-serif;'\
           'text-align:center;margin-top:50px;font-size:24;}input{marg
in-top:20px;margin-left:10px;margin-right:10px;}'\
           '</style></head><body>Enter new password<form action="/acco
unt/change_password" method="post"><input type="password" name="passwo
rd">'\
           '<input type="submit" value="Confirm"><input type="hidden" 
name="token" value="' + token + '"></form></body></html>'
def get_db():
    db = getattr(g, '_database', None)
    if db is None:
        db = g._database = sqlite3.connect(DATABASE_PATH)
        db.execute('PRAGMA foreign_keys = ON')
    return db
\end{verbatim}

\subsection{bluetooth.py}
\begin{verbatim}
from flask import Blueprint, request
import base
from util import error, get_db, log_info, ok
mod = Blueprint('bluetooth', __name__)
@mod.route('/bluetooth/user_about')
def user_about():
    args = request.args
    bid = args.get('bid')
    user = base.get_user_by_bid(bid)
    if not user:
        return error('invalid bid', 400)
    log_info(f'User about {user.email}')
    return user.about()
@mod.route('/bluetooth/set_bid', methods=['POST'])
def set_bid():
    args = request.args
    token = args.get('token')
    bid = args.get('bid')
    if len(bid) == 0:
        return error('invalid bid', 400)
    user = base.get_user_by_token(token, 'access')
    if not user:
        return error('invalid token', 401)
    base.set_user_bid(user, bid)
    get_db().commit()
    log_info(f'Set bid: {user.email}, {bid}')
    return ok()
\end{verbatim}

\subsection{messenger.py}
\begin{verbatim}
import json
from flask import Blueprint, request
import base
from models.messenger import MessageModel
from util import error, get_db, log_info, ok
mod = Blueprint('messenger', __name__)
@mod.route('/messenger/chats')
def chats():
    args = request.args
    token = args.get('token')
    user = base.get_user_by_token(token, 'access')
    if not user:
        return error('invalid token', 401)
    chats = base.get_chats(user)
    for chat in chats:
        if chat['type'] == 'private':
            recipients = [r for r in chat['users'] if r['id'] != user.
id]
            chat['name'] = recipients[0]['email'] if len(recipients) >
 0 else 'Account deleted'
    log_info(f'Get chats for user {user.email}')
    return {'chats': chats}
@mod.route('/messenger/chat_history')
def chat_history():
    args = request.args
    token = args.get('token')
    user = base.get_user_by_token(token, 'access')
    if not user:
        return error('invalid token', 401)
    chat_id = args.get('chat_id')
    if chat_id is None:
        return error('invalid chat id', 400)
    if not base.is_user_in_chat(user, chat_id):
        return error('invalid chat id', 400)
    chat = base.get_chat_history(chat_id)
    if chat['type'] == 'private':
        recipients = [r for r in chat['users'] if r['id'] != user.id]
        chat['name'] = recipients[0]['email'] if len(recipients) > 0 e
lse 'Account deleted'
    log_info(f'Get chat history for user {user.email}')
    return chat
@mod.route('/messenger/start_dialog', methods=['POST'])
def start_dialog():
    args = request.args
    content = request.json
    token = args.get('token')
    user = base.get_user_by_token(token, 'access')
    if not user:
        return error('invalid token', 401)
    recipient_id = args.get('user_id')
    if recipient_id is None:
        return error('invalid user id', 400)
    recipient = base.get_user_by_id(recipient_id)
    if recipient is None:
        return error('invalid user id', 400)
    if recipient.id == user.id:
        return error('self chats are not supported', 400)
    message = MessageModel(content)
    if not message.valid():
        return error('invalid message', 405)
    log_info(f'Start dialog by user {user.email} for user {recipient.e
mail}')
    chat_id = base.get_private_chat_id(user, recipient)
    if not chat_id:
        return error('chat already started', 402)
    base.send_message(user, chat_id, message)
    get_db().commit()
    return {'chat_id': chat_id}
@mod.route('/messenger/user_search')
def user_search():
    args = request.args
    token = args.get('token')
    user = base.get_user_by_token(token, 'access')
    if not user:
        return error('invalid token', 401)
    email = args.get('email')
    if email is None:
        return error('invalid email to search', 400)
    log_info(f'Search by user {user.email} for {email}')
    return {'users': base.user_search(user, email)}
@mod.route('/messenger/create_chat', methods=['POST'])
def create_chat():
    args = request.args
    content = request.json
    token = args.get('token')
    user = base.get_user_by_token(token, 'access')
    if not user:
        return error('invalid token', 401)
    chat_name = args.get('name')
    if type(chat_name) is not str or len(chat_name) == 0:
        return error('invalid chat name', 402)
    user_ids = content.get('users', None)
    if type(user_ids) is not list or len(user_ids) == 0:
        return error('invalid users list', 405)
    users = []
    for user_id in user_ids:
        current_user = base.get_user_by_id(user_id)
        if current_user is None:
            return error('invalid user id in list', 402)
        users.append(current_user)
    chat_id = base.create_chat(user, users, chat_name)
    get_db().commit()
    log_info(f'Create public chat by user {user.email}')
    return {'chat_id': chat_id}
@mod.route('/messenger/send_message', methods=['POST'])
def send_message():
    args = request.args
    content = request.json
    token = args.get('token')
    user = base.get_user_by_token(token, 'access')
    if not user:
        return error('invalid token', 401)
    chat_id = args.get('chat_id')
    if not base.is_user_in_chat(user, chat_id):
        return error('invalid chat id', 400)
    message = MessageModel(content)
    if not message.valid():
        return error('invalid message', 405)
    base.send_message(user, chat_id, message)
    get_db().commit()
    log_info(f'Send message {message.text} by user {user.email}')
    return ok()
@mod.route('/messenger/delete_message', methods=['POST'])
def delete_message():
    args = request.args
    token = args.get('token')
    user = base.get_user_by_token(token, 'access')
    if not user:
        return error('invalid token', 401)
    message_id = args.get('message_id')
    message = base.get_message_by_id(message_id)
    if not message or message.user_id != user.id:
        return error('invalid message id', 400)
    base.delete_message(message)
    get_db().commit()
    log_info(f'Deleted message {message.text} by user {user.email}')
    return ok()
@mod.route('/messenger/delete_chat', methods=['POST'])
def delete_chat():
    args = request.args
    token = args.get('token')
    user = base.get_user_by_token(token, 'access')
    if not user:
        return error('invalid token', 401)
    chat_id = args.get('chat_id')
    if not base.is_user_in_chat(user, chat_id):
        return error('invalid chat id', 400)
    chat = base.get_chat_by_id(chat_id)
    if chat.type == 'private':
        base.delete_chat(chat)
        log_info(f'Deleted private chat {chat_id} by user {user.email}
')
    else:
        base.delete_membership(user, chat)
        log_info(f'Deleted membership in chat {chat_id} by user {user.
email}')
    get_db().commit()
    return ok()
\end{verbatim}

\subsection{account.py}
\begin{verbatim}
from flask import Blueprint, request
import base
from models.account import Account, UserAbout, User, check_email, chec
k_password
from secrets import secrets
from util import error, get_db, hash_password, log_info, ok, send_emai
l, html_message, html_password_change
mod = Blueprint('account', __name__)
@mod.route('/account/create', methods=['POST'])
def create():
    content = request.json
    account = Account(content)
    if not account.valid():
        return error('invalid email or password', 400)
    user = base.get_user_by_email(account.email)
    if user:
        return error('email already registered', 405)
    user = base.create_user(account)
    access_token = base.create_token(user, 'access')
    confirm_token = base.create_token(user, 'confirm')
    get_db().commit()
    send_email(user.email, 'Account verification', 'http://' + secrets
['host_address'] + '/account/confirm?token=' + confirm_token)
    log_info(f'User created {user.email}')
    return {'token': access_token}
@mod.route('/account/confirm', methods=['GET'])
def confirm():
    args = request.args
    confirm_token = args.get('token')
    user = base.get_user_by_token(confirm_token, 'confirm')
    title = 'Account verification'
    if not user:
        return html_message(title, 'Invalid verification token')
    if user.confirmed:
        return html_message(title, 'Account is already verified')
    base.set_user_confirmed(user, 1)
    get_db().commit()
    log_info(f'User confirmed {user.email}')
    return html_message(title, 'Account has been verified')
@mod.route('/account/recovery', methods=['POST'])
def recovery():
    args = request.args
    email = args.get('email')
    if not check_email(email):
        return error('invalid email', 400)
    user = base.get_user_by_email(email)
    if not user:
        return error('email not registered', 405)
    recovery_token = base.create_token(user, 'recovery')
    get_db().commit()
    send_email(user.email, 'Account recovery', 'http://' + secrets['ho
st_address'] + '/account/recovery_form?token=' + recovery_token)
    log_info(f'Password recovery request {email}')
    return ok()
@mod.route('/account/recovery_form', methods=['GET'])
def recovery_form():
    args = request.args
    recovery_token = args.get('token')
    user = base.get_user_by_token(recovery_token, 'recovery')
    if not user:
        return html_message('Account recovery', 'Invalid recovery toke
n')
    log_info(f'Password recovery form request {user.email}')
    return html_password_change(recovery_token)
@mod.route('/account/change_password', methods=['POST'])
def change_password():
    content = request.form
    recovery_token = content['token']
    password = content['password']
    title = 'Account recovery'
    if not check_password(password):
        return html_message(title, 'Invalid password')
    password = hash_password(password)
    user = base.get_user_by_token(recovery_token, 'recovery')
    if not user:
        return html_message(title, 'Invalid recovery token')
    base.set_user_password(user, password)
    get_db().commit()
    log_info(f'User password changed {user.email}')
    return html_message(title, 'Password has been changed')
@mod.route('/account/auth', methods=['POST'])
def auth():
    content = request.json
    account = Account(content)
    if not account.valid():
        return error('invalid email or password', 400)
    user = base.get_user_by_email(account.email)
    if not user:
        return error('email not registered', 405)
    if not user.match(account):
        return error('wrong password', 406)
    token = base.create_token(user, 'access')
    get_db().commit()
    log_info(f'Authentication {user.email}')
    return {'token': token}
@mod.route('/account/delete', methods=['POST'])
def delete():
    args = request.args
    token = args.get('token')
    user = base.get_user_by_token(token, 'access')
    if not user:
        return error('invalid token', 401)
    base.delete_user(user)
    get_db().commit()
    log_info(f'User deleted {user.email}')
    return ok()
@mod.route('/account/set_about', methods=['POST'])
def set_about():
    args = request.args
    content = request.json
    token = args.get('token')
    about = UserAbout(content)
    if not about.valid():
        return error('invalid about section', 405)
    user = base.get_user_by_token(token, 'access')
    if not user:
        return error('invalid token', 401)
    base.set_user_about(user, about)
    get_db().commit()
    log_info(f'User about updated {user.email}')
    return ok()
\end{verbatim}

\subsection{messenger.py}
\begin{verbatim}
class Chat:
    def __init__(self, row):
        self.id = row[0]
        self.type = row[1]
        self.name = row[2]
class MessageModel:
    def __init__(self, json):
        self.text = json.get('text', None)
        self.time = json.get('time', None)
    def valid(self):
        return type(self.text) == str and len(self.text) > 0
class Message:
    def __init__(self, row):
        self.id = row[0]
        self.chat_id = row[1]
        self.user_id = row[2]
        self.time = row[3]
        self.text = row[4]
\end{verbatim}

\subsection{account.py}
\begin{verbatim}
import re
from constants import MIN_PASSWORD_LENGTH
from util import hash_password
def check_email(email):
    return email != None and re.match(r'\S+@\S+\.\S+', email) != None
def check_password(password):
    return password != None and len(password) >= MIN_PASSWORD_LENGTH
def check_name(name):
    return name != None and len(name) > 0
class Account:
    def __init__(self, json):
        self.email = json.get('email', None)
        self.password = json.get('password', None)
        if check_password(self.password):
            self.password = hash_password(self.password)
    def valid(self):
        return check_email(self.email) and check_password(self.passwor
d)
class UserAbout:
    def __init__(self, json):
        self.name = json.get('name', None)
    def valid(self):
        return check_name(self.name)
class User:
    def __init__(self, row):
        self.id = row[0]
        self.email = row[1]
        self.password = row[2]
        self.confirmed = row[3]
        self.name = row[4]
        self.bid = row[5]
    def match(self, account):
        return self.email == account.email and self.password == accoun
t.password
    def about(self):
        return {
            'id': self.id,
            'email': self.email,
            'confirmed': self.confirmed,
            'about': {
                'name': self.name
            }
        }
class Token:
    def __init__(self, row):
        self.id = row[0]
        self.type = row[1]
        self.token = row[2]
        self.user_id = row[3]
\end{verbatim}

\section{Исходный код миграций}

\subsection{v01\_init.sql}
\begin{verbatim}
CREATE TABLE user (
    id INTEGER PRIMARY KEY AUTOINCREMENT,
    email TEXT NOT NULL,
    password TEXT NOT NULL,
    confirmed INTEGER DEFAULT 0,
    name TEXT,
    bid TEXT
);
CREATE TABLE token (
    id INTEGER PRIMARY KEY AUTOINCREMENT,
    type TEXT NOT NULL,
    token TEXT NOT NULL,
    user_id INTEGER NOT NULL,
    FOREIGN KEY(user_id) REFERENCES user(id)
        ON DELETE CASCADE
);
CREATE TABLE chat (
    id INTEGER PRIMARY KEY AUTOINCREMENT,
    type TEXT NOT NULL,
    name TEXT
);
CREATE TABLE membership (
    user_id INTEGER NOT NULL,
    chat_id INTEGER NOT NULL,
    FOREIGN KEY(user_id) REFERENCES user(id)
        ON DELETE CASCADE,
    FOREIGN KEY(chat_id) REFERENCES chat(id)
        ON DELETE CASCADE
);
CREATE TABLE message (
    id INTEGER PRIMARY KEY AUTOINCREMENT,
    chat_id INTEGER NOT NULL,
    user_id INTEGER NOT NULL,
    time TEXT NOT NULL,
    text TEXT NOT NULL,
    FOREIGN KEY(chat_id) REFERENCES chat(id)
        ON DELETE CASCADE
);
\end{verbatim}


\bibliography{espd,library}

\end{document}
